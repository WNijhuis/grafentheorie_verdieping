\documentclass[11pt,a4paper]{article}
\usepackage{blindtext}
\usepackage{multicol}
\usepackage[T1]{fontenc}
\usepackage[utf8]{inputenc}
\usepackage{lmodern}
\usepackage[dutch]{babel}
\usepackage{amsmath,amssymb,amsthm}
\usepackage{graphicx, csquotes}
\usepackage[style=numeric]{biblatex}
\usepackage[a4paper,
  lmargin=0.1666\paperwidth,
  rmargin=0.1666\paperwidth,
  tmargin=0.1111\paperheight,
  bmargin=0.1111\paperheight]{geometry}
\linespread{1.2}
\newcommand{\set}[1]{\mathbb{#1}}
\newcommand{\abs}[1]{\left\lvert #1 \right\rvert}
\newcommand{\floor}[1]{\left\lfloor #1 \right\rfloor}
\newcommand{\powerset}[1]{\mathcal{P}\left(#1\right)}
\newcommand{\dd}{\mathrm{d}}

\newenvironment{exercise}[1]{
  \par\noindent
  \textbf{Opgave #1.~}
  \itshape
}{
  \par
  \upshape
}

\begin{document}

\begin{titlepage}
    \centering
    {\Huge \textbf{Ontwikkelingen in de Ramsey-theorie}} \\[0.5cm]
    {\huge Verdiepingsopdracht grafentheorie} \\[1cm]
    {\Large Merijn Post, Timo Boomsma, Ward Nijhuis} \\[0.5cm]
    {\Large 16127382, 16239288, 15956547} \\[0.5cm]
    {\Large \today} \\[0.5cm]
    {\Large dr. Ross Kang} \\[1cm]
    \includegraphics[width=1\textwidth]{uva.jpg} \\[1cm]
\end{titlepage}

\begin{abstract}

\end{abstract}

\begin{multicols}{2}
    \section{Inleiding}
    Hoeveel mensen moet je uitnodigen voor een feestje zodat er of drie mensen zijn die elkaar allemaal kennen of drie mensen die elkaar allemaal niet kennen? Het antwoord op dit vraagstuk en vele vergelijkbare problemen komt uit de zogeheten Ramsey-theorie, een deelgebied van de grafentheorie die begin 20e eeuw is bedacht door de Britse wiskundige Frank Ramsey.
    \section{Geschiedenis}
    \section{Paul Erdős}
    \section{R(5,5)}
    \section{Conclusie}
\end{multicols}

\end{document}